% Options for packages loaded elsewhere
\PassOptionsToPackage{unicode}{hyperref}
\PassOptionsToPackage{hyphens}{url}
%
\documentclass[
]{ctexart}
\usepackage{amsmath,amssymb}
\usepackage{iftex}
\ifPDFTeX
  \usepackage[T1]{fontenc}
  \usepackage[utf8]{inputenc}
  \usepackage{textcomp} % provide euro and other symbols
\else % if luatex or xetex
  \usepackage{unicode-math} % this also loads fontspec
  \defaultfontfeatures{Scale=MatchLowercase}
  \defaultfontfeatures[\rmfamily]{Ligatures=TeX,Scale=1}
\fi
\usepackage{lmodern}
\ifPDFTeX\else
  % xetex/luatex font selection
\fi
% Use upquote if available, for straight quotes in verbatim environments
\IfFileExists{upquote.sty}{\usepackage{upquote}}{}
\IfFileExists{microtype.sty}{% use microtype if available
  \usepackage[]{microtype}
  \UseMicrotypeSet[protrusion]{basicmath} % disable protrusion for tt fonts
}{}
\makeatletter
\@ifundefined{KOMAClassName}{% if non-KOMA class
  \IfFileExists{parskip.sty}{%
    \usepackage{parskip}
  }{% else
    \setlength{\parindent}{0pt}
    \setlength{\parskip}{6pt plus 2pt minus 1pt}}
}{% if KOMA class
  \KOMAoptions{parskip=half}}
\makeatother
\usepackage{xcolor}
\usepackage[margin=2cm]{geometry}
\usepackage{color}
\usepackage{fancyvrb}
\newcommand{\VerbBar}{|}
\newcommand{\VERB}{\Verb[commandchars=\\\{\}]}
\DefineVerbatimEnvironment{Highlighting}{Verbatim}{commandchars=\\\{\}}
% Add ',fontsize=\small' for more characters per line
\usepackage{framed}
\definecolor{shadecolor}{RGB}{248,248,248}
\newenvironment{Shaded}{\begin{snugshade}}{\end{snugshade}}
\newcommand{\AlertTok}[1]{\textcolor[rgb]{0.94,0.16,0.16}{#1}}
\newcommand{\AnnotationTok}[1]{\textcolor[rgb]{0.56,0.35,0.01}{\textbf{\textit{#1}}}}
\newcommand{\AttributeTok}[1]{\textcolor[rgb]{0.13,0.29,0.53}{#1}}
\newcommand{\BaseNTok}[1]{\textcolor[rgb]{0.00,0.00,0.81}{#1}}
\newcommand{\BuiltInTok}[1]{#1}
\newcommand{\CharTok}[1]{\textcolor[rgb]{0.31,0.60,0.02}{#1}}
\newcommand{\CommentTok}[1]{\textcolor[rgb]{0.56,0.35,0.01}{\textit{#1}}}
\newcommand{\CommentVarTok}[1]{\textcolor[rgb]{0.56,0.35,0.01}{\textbf{\textit{#1}}}}
\newcommand{\ConstantTok}[1]{\textcolor[rgb]{0.56,0.35,0.01}{#1}}
\newcommand{\ControlFlowTok}[1]{\textcolor[rgb]{0.13,0.29,0.53}{\textbf{#1}}}
\newcommand{\DataTypeTok}[1]{\textcolor[rgb]{0.13,0.29,0.53}{#1}}
\newcommand{\DecValTok}[1]{\textcolor[rgb]{0.00,0.00,0.81}{#1}}
\newcommand{\DocumentationTok}[1]{\textcolor[rgb]{0.56,0.35,0.01}{\textbf{\textit{#1}}}}
\newcommand{\ErrorTok}[1]{\textcolor[rgb]{0.64,0.00,0.00}{\textbf{#1}}}
\newcommand{\ExtensionTok}[1]{#1}
\newcommand{\FloatTok}[1]{\textcolor[rgb]{0.00,0.00,0.81}{#1}}
\newcommand{\FunctionTok}[1]{\textcolor[rgb]{0.13,0.29,0.53}{\textbf{#1}}}
\newcommand{\ImportTok}[1]{#1}
\newcommand{\InformationTok}[1]{\textcolor[rgb]{0.56,0.35,0.01}{\textbf{\textit{#1}}}}
\newcommand{\KeywordTok}[1]{\textcolor[rgb]{0.13,0.29,0.53}{\textbf{#1}}}
\newcommand{\NormalTok}[1]{#1}
\newcommand{\OperatorTok}[1]{\textcolor[rgb]{0.81,0.36,0.00}{\textbf{#1}}}
\newcommand{\OtherTok}[1]{\textcolor[rgb]{0.56,0.35,0.01}{#1}}
\newcommand{\PreprocessorTok}[1]{\textcolor[rgb]{0.56,0.35,0.01}{\textit{#1}}}
\newcommand{\RegionMarkerTok}[1]{#1}
\newcommand{\SpecialCharTok}[1]{\textcolor[rgb]{0.81,0.36,0.00}{\textbf{#1}}}
\newcommand{\SpecialStringTok}[1]{\textcolor[rgb]{0.31,0.60,0.02}{#1}}
\newcommand{\StringTok}[1]{\textcolor[rgb]{0.31,0.60,0.02}{#1}}
\newcommand{\VariableTok}[1]{\textcolor[rgb]{0.00,0.00,0.00}{#1}}
\newcommand{\VerbatimStringTok}[1]{\textcolor[rgb]{0.31,0.60,0.02}{#1}}
\newcommand{\WarningTok}[1]{\textcolor[rgb]{0.56,0.35,0.01}{\textbf{\textit{#1}}}}
\usepackage{graphicx}
\makeatletter
\def\maxwidth{\ifdim\Gin@nat@width>\linewidth\linewidth\else\Gin@nat@width\fi}
\def\maxheight{\ifdim\Gin@nat@height>\textheight\textheight\else\Gin@nat@height\fi}
\makeatother
% Scale images if necessary, so that they will not overflow the page
% margins by default, and it is still possible to overwrite the defaults
% using explicit options in \includegraphics[width, height, ...]{}
\setkeys{Gin}{width=\maxwidth,height=\maxheight,keepaspectratio}
% Set default figure placement to htbp
\makeatletter
\def\fps@figure{htbp}
\makeatother
\setlength{\emergencystretch}{3em} % prevent overfull lines
\providecommand{\tightlist}{%
  \setlength{\itemsep}{0pt}\setlength{\parskip}{0pt}}
\setcounter{secnumdepth}{-\maxdimen} % remove section numbering
\ifLuaTeX
  \usepackage{selnolig}  % disable illegal ligatures
\fi
\usepackage{bookmark}
\IfFileExists{xurl.sty}{\usepackage{xurl}}{} % add URL line breaks if available
\urlstyle{same}
\hypersetup{
  pdftitle={01},
  hidelinks,
  pdfcreator={LaTeX via pandoc}}

\title{01}
\author{}
\date{\vspace{-2.5em}}

\begin{document}
\maketitle

{
\setcounter{tocdepth}{2}
\tableofcontents
}
\subsection{导入数据}\label{ux5bfcux5165ux6570ux636e}

\begin{Shaded}
\begin{Highlighting}[]
\FunctionTok{library}\NormalTok{(tidyverse)}
\NormalTok{data }\OtherTok{\textless{}{-}} \FunctionTok{data.frame}\NormalTok{(}
  \AttributeTok{x1 =} \FunctionTok{c}\NormalTok{(}\FloatTok{3.7}\NormalTok{,}\FloatTok{5.7}\NormalTok{,}\FloatTok{3.8}\NormalTok{,}\FloatTok{3.2}\NormalTok{,}\FloatTok{3.1}\NormalTok{,}\FloatTok{4.6}\NormalTok{,}\FloatTok{2.4}\NormalTok{,}\FloatTok{7.2}\NormalTok{,}\FloatTok{6.7}\NormalTok{,}\FloatTok{5.4}\NormalTok{,}\FloatTok{3.9}\NormalTok{,}\FloatTok{4.5}\NormalTok{,}\FloatTok{3.5}\NormalTok{,}\FloatTok{4.5}\NormalTok{,}\FloatTok{1.5}\NormalTok{,}\FloatTok{8.5}\NormalTok{,}\FloatTok{4.5}\NormalTok{,}\FloatTok{6.5}\NormalTok{,}\FloatTok{4.1}\NormalTok{,}\FloatTok{5.5}\NormalTok{),}
  \AttributeTok{x2 =} \FunctionTok{c}\NormalTok{(}\FloatTok{48.5}\NormalTok{,}\FloatTok{65.1}\NormalTok{,}\FloatTok{47.2}\NormalTok{,}\FloatTok{53.2}\NormalTok{,}\FloatTok{55.5}\NormalTok{,}\FloatTok{36.1}\NormalTok{,}\FloatTok{24.8}\NormalTok{,}\FloatTok{33.1}\NormalTok{,}\FloatTok{47.4}\NormalTok{,}\FloatTok{54.1}\NormalTok{,}\FloatTok{36.9}\NormalTok{,}\FloatTok{58.8}\NormalTok{,}\FloatTok{27.8}\NormalTok{,}\FloatTok{40.2}\NormalTok{,}\FloatTok{13.5}\NormalTok{,}\FloatTok{56.4}\NormalTok{,}\FloatTok{71.6}\NormalTok{,}\FloatTok{52.8}\NormalTok{,}\FloatTok{44.1}\NormalTok{,}\FloatTok{40.9}\NormalTok{),}
  \AttributeTok{x3 =} \FunctionTok{c}\NormalTok{(}\FloatTok{9.3}\NormalTok{,}\FloatTok{8.0}\NormalTok{,}\FloatTok{10.9}\NormalTok{,}\FloatTok{12.0}\NormalTok{,}\FloatTok{9.7}\NormalTok{,}\FloatTok{7.9}\NormalTok{,}\FloatTok{14.0}\NormalTok{,}\FloatTok{7.6}\NormalTok{,}\FloatTok{8.5}\NormalTok{,}\FloatTok{11.3}\NormalTok{,}\FloatTok{12.7}\NormalTok{,}\FloatTok{12.3}\NormalTok{,}\FloatTok{9.8}\NormalTok{,}\FloatTok{8.4}\NormalTok{,}\FloatTok{10.1}\NormalTok{,}\FloatTok{7.1}\NormalTok{,}\FloatTok{8.2}\NormalTok{,}\FloatTok{10.9}\NormalTok{,}\FloatTok{11.2}\NormalTok{,}\FloatTok{9.4}\NormalTok{)}
\NormalTok{)}
\NormalTok{data }\SpecialCharTok{\%\textgreater{}\%} \FunctionTok{head}\NormalTok{(}\DecValTok{5}\NormalTok{)}
\end{Highlighting}
\end{Shaded}

\begin{verbatim}
##    x1   x2   x3
## 1 3.7 48.5  9.3
## 2 5.7 65.1  8.0
## 3 3.8 47.2 10.9
## 4 3.2 53.2 12.0
## 5 3.1 55.5  9.7
\end{verbatim}

\subsection{第(1)问}\label{ux7b2c1ux95ee}

\begin{Shaded}
\begin{Highlighting}[]
\FunctionTok{library}\NormalTok{(ICSNP)}
\NormalTok{mu0 }\OtherTok{\textless{}{-}} \FunctionTok{c}\NormalTok{(}\DecValTok{4}\NormalTok{, }\DecValTok{50}\NormalTok{, }\DecValTok{10}\NormalTok{)}
\NormalTok{result }\OtherTok{\textless{}{-}} \FunctionTok{HotellingsT2}\NormalTok{(data, }\AttributeTok{mu=}\NormalTok{mu0)}
\FunctionTok{print}\NormalTok{(result)}
\end{Highlighting}
\end{Shaded}

\begin{verbatim}
## 
##  Hotelling's one sample T2-test
## 
## data:  data
## T.2 = 2.9045, df1 = 3, df2 = 17, p-value = 0.06493
## alternative hypothesis: true location is not equal to c(4,50,10)
\end{verbatim}

p值0.06493,不能拒绝原假设

\subsection{第(2)问}\label{ux7b2c2ux95ee}

\begin{Shaded}
\begin{Highlighting}[]
\CommentTok{\# 计算样本均值向量和协方差矩阵}
\NormalTok{x\_bar }\OtherTok{\textless{}{-}} \FunctionTok{colMeans}\NormalTok{(data)}
\NormalTok{S }\OtherTok{\textless{}{-}} \FunctionTok{cov}\NormalTok{(data)}
\NormalTok{S\_inv }\OtherTok{\textless{}{-}} \FunctionTok{solve}\NormalTok{(S)}

\NormalTok{n }\OtherTok{\textless{}{-}} \FunctionTok{nrow}\NormalTok{(data)  }\CommentTok{\# 样本量 n = 20}
\NormalTok{p }\OtherTok{\textless{}{-}} \FunctionTok{ncol}\NormalTok{(data)  }\CommentTok{\# 变量维度 p = 3}

\CommentTok{\# 计算临界值}
\NormalTok{alpha }\OtherTok{\textless{}{-}} \FloatTok{0.05}
\NormalTok{F\_critical }\OtherTok{\textless{}{-}} \FunctionTok{qf}\NormalTok{(}\DecValTok{1} \SpecialCharTok{{-}}\NormalTok{ alpha, p, n }\SpecialCharTok{{-}}\NormalTok{ p)}
\NormalTok{c\_value }\OtherTok{\textless{}{-}}\NormalTok{ (p }\SpecialCharTok{*}\NormalTok{ (n }\SpecialCharTok{{-}} \DecValTok{1}\NormalTok{)) }\SpecialCharTok{/}\NormalTok{ (n }\SpecialCharTok{*}\NormalTok{ (n }\SpecialCharTok{{-}}\NormalTok{ p)) }\SpecialCharTok{*}\NormalTok{ F\_critical}

\CommentTok{\# 输出结果}
\FunctionTok{cat}\NormalTok{(}\StringTok{"样本均值向量 x\_bar:}\SpecialCharTok{\textbackslash{}n}\StringTok{"}\NormalTok{)}
\end{Highlighting}
\end{Shaded}

\begin{verbatim}
## 样本均值向量 x_bar:
\end{verbatim}

\begin{Shaded}
\begin{Highlighting}[]
\FunctionTok{print}\NormalTok{(x\_bar)}
\end{Highlighting}
\end{Shaded}

\begin{verbatim}
##     x1     x2     x3 
##  4.640 45.400  9.965
\end{verbatim}

\begin{Shaded}
\begin{Highlighting}[]
\FunctionTok{cat}\NormalTok{(}\StringTok{"}\SpecialCharTok{\textbackslash{}n}\StringTok{协方差矩阵的逆 S\^{}\{{-}1\}:}\SpecialCharTok{\textbackslash{}n}\StringTok{"}\NormalTok{)}
\end{Highlighting}
\end{Shaded}

\begin{verbatim}
## 
## 协方差矩阵的逆 S^{-1}:
\end{verbatim}

\begin{Shaded}
\begin{Highlighting}[]
\FunctionTok{print}\NormalTok{(S\_inv)}
\end{Highlighting}
\end{Shaded}

\begin{verbatim}
##             x1           x2           x3
## x1  0.58615531 -0.022085719  0.257968742
## x2 -0.02208572  0.006067227 -0.001580929
## x3  0.25796874 -0.001580929  0.401846765
\end{verbatim}

\begin{Shaded}
\begin{Highlighting}[]
\FunctionTok{cat}\NormalTok{(}\StringTok{"}\SpecialCharTok{\textbackslash{}n}\StringTok{椭球方程的临界值 c ="}\NormalTok{, c\_value)}
\end{Highlighting}
\end{Shaded}

\begin{verbatim}
## 
## 椭球方程的临界值 c = 0.5359302
\end{verbatim}

样本均值向量 \(\overline{x}=(4.615,45.305,9.965)'\)

协方差矩阵的逆 \(S^{-1}=\left(\)

\end{document}
